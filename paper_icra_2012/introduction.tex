\section{Introduction} \label{sec:introduction}

% measurement noise model

Autonomous robots must be able to operate in the real world, perceiving the
environment using their sensors and acting accordingly. In order to accomplish
this, they must deal with the uncertainty that is inherent to real environments.
As discussed by Thrun et al. in \cite{Thrun:2005:PR:1121596}, there are several
factors that contribute to a robot's uncertainty: environments are unpredictable
and partially observable; sensors and actuators are noisy and faulty; and models
of the environment and robot are approximated. To cope with various sources of
uncertainty, probabilistic methods have been successfully used in several
robotic applications such as localization, mapping and decision making.

One of the key components of probabilistic methods is a good sensor model. This
model represents the process by which the sensor measurements are generated. A
good sensor model can adequately explain the noise that is inherent in the
observed measurements. Several sensors, and their respective sensor models, have
been studied in the literature, ranging from classical odometry sensors
\cite{Borenstein95}, cameras \cite{Wu07}\cite{clemente2007mapping}, lasers
\cite{newman2009navigating}, to \ac{TOF} cameras \cite{may2009three}.

In the last couple of years a novel sensor gained attention in the robotics
community. The comercialization of the Kinect\texttrademark ~ by Microsoft
brought with it an inexpensive depth sensor that uses an active range system to
generate a depth map of a given environment \cite{Freedman2008}. The
Kinect-style sensors provide RGB-D images which include both visual (RGB) and
depth (D) values at 30 frames per second and 640$\times$480 resolution. Several
works have taken advantage of this sensor technology in scenarios such as
environmental mapping \cite{henry2012rgb}, 3D reconstruction
\cite{Newcombe2011}, gesture recognition \cite{Xia2011}, and altitude control of
aerial vehicles \cite{Stowers2011}.
%\cite{Newcombe2011,Xia2011,Stowers2011,henry2012rgb}.

% In the last couple of years a novel sensor gained attention in the robotics community.  RGB-D sensors, like the Microsoft Kinect, are able to provide both visual (RGB) and depth (D) information in an environment.  The Kinect sensor in particular, uses an active stereo system to generate a depth map of the environment {REF}.  Several works have taken advantage of this pairing of information in scenarios such as environmental mapping, 3D reconstruction , human action detection, and altitude control of aerial vehicles {REF}.

% The objective of this paper is to develop a measurement noise model for the Kinect sensor. From a statistical analysis of a large number of experiments, we are able to provide a function relating the orientation and distance of a point as provided by the Microsoft Kinect to a reliability value. Two main works have created ad hoc models for the Kinect \cite{Newcombe2011,Fallon2012}. The contribution of this work is a measurement noise model which is determined from actual experimentation.

The objective of this paper is to develop a measurement noise model for
Kinect-style sensors. Differently from previous works, which have created ad hoc
models for these types of sensors \cite{Newcombe2011}\cite{Fallon2012}, we
performed a statistical analysis on a large data set to provide a value for
uncertainty as a function of distance and angle of incidence of a measurement.

The rest of the paper is laid out in the following manner: Section
\ref{sec:related_works} presents previous works related to the generation and
verification of measurement noise models for depth sensors, especially those
related to Kinect-style sensors. Section \ref{sec:approach} discusses the
approach used for generating the model, including both the theory behind the
methods used as well as how they were implemented. Next, Section
\ref{sec:results} presents the results of our methodology and gives the model
generated for the sensor. In this section, we also validate our model with a
second Kinect and an ASUS Xtion PRO. Finally, Section
\ref{sec:conclusion} summarizes our results and the usefulness of our model.


%The rest of the paper is laid out in the following manner:
%Section II presents previous works related to the generation and verification of noise models for depth sensors, specifically those related to the Kinect sensor.
%Section III discusses the approach used in generating the model, both the theory behind the methods used as well as how they were implemented.
%Next, Section IV gives the results of the proposed methodology as well as at the sensor model that was generated.  This section also presents a validation of the model using a different Kinect as well as the ASUS Xtion PRO sensor.
%Finally, Section V outlines the results of this work as well as the usefulness and implementation of the proposed model.
