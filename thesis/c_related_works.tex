\chapter{Related Works}	\label{chapter:related_works}

% Works related to MABDI are generally based on RGB-D sensors. This type of sensor has
% become very popular since the release of the Kinect from Microsoft, which
% was the first mass produced RGB-D sensor of its kind. RGB-D sensors are
% inexpensive and produce noisy 640x480 depth images at 30fps. The RGB-D
% sensor has excited the robotics community because this has been the first
% time that depth data has been so readily accessible from such an
% inexpensive sensor. Therefore, methodologies that use RGB-D data must be able to quickly
% deal with very high rates of information.
%
% Research and development of new mapping algorithms trend towards
% leveraging the information in the global map to make decisions about the
% incoming data. One can see parallels with how we as humans see the world. MABDI
% proposes do this in a computationally feasible way by simply using
% differencing and thresholding imaging methods.


A major problem in robotics has been and continues to be: How can we create the
``best'' representation of an unknown environment? There are two main
communities of researchers who been working on developing algorithms and methods
to answer precisely this question. They are the robotics community and the
computer graphics community, but each community has a slightly different
motivation for solving this problem. The robotics community is concerned with
developing a real-time solution for generating representations in large
environments. In the literature, large environments usually range in size from
multi-room office spaces to a few square miles on city streets. These
representations are used by both fully autonomous and teleoperated systems. The
common name which is used by the robotics community for this problem is
Simultaneous Localization and Mapping or SLAM. The name SLAM refers to the
problem of mapping and locating a robot in an unknown environment. Early methods
generated very sparse representations of the world, as time and sensor
technology progressed the representations became denser. A dense representation
is desired for any system that must have good situational awareness of its
environment. The computer graphics community is concerned with generating high
quality representations of small environments. In the literature, small
environments usually range in size from a cubic meter to room size. They
generally refer to the problem as surface reconstruction. These representations
are used by augmented reality, computer game object creation, 3D printing, and
other applications. In the following sections we will trace the development of
representation generating methods in both communities.

\section{SLAM}

The problem of SLAM has been a primary focus of the robotics community for
more than 25 years. A complete solution to the SLAM problem must be able to
generate a representation of an unknown environment and track the robot in
this new representation. In this body of literature the act of generating a
representation is referred to as mapping. A good overview of the problem
can be found in \cite{Durrant-Whyte2006} and \cite{Bailey2006}. Each
solution is designed to consider the goal application, type of sensor,
computational constraints, and memory limits. All these factors influence
the researcher's choice of which type of representation to use for the
mapping procedure. In 2002 Thrun wrote a famous survey \cite{Thrun2002} of
the SLAM literature which categorized existing algorithms on many traits
including the representation. The representation choice of prior work can
be roughly categorized into three types. The first type is characterized by
some sort of list of 2D or 3D points and are usually considered to be
sparse representations. Common names for these types are landmark locations
and point clouds. The second type is considered to be more volumetric
based and is often times considered to be a dense representation. Common
names for these types are occupancy grid and Truncated Signed Distance
Function (TSDF). The last type has the characteristic of being a surface
representation and is also considered to be a dense representation. Common
names for these types are surfels and mesh. In the following sections we
will trace the history of each of the three types of representation that are
seen in the SLAM literature.

\subsection{Point Locations}

One of the most well known and earliest solution to the SLAM problem, which
uses a point location representation, was proposed by Smith et al. in 1990
\cite{Smith1990}. The mathematical framework that he created was the origin
of a family of solutions based on the Extended Kalman Filter (EKF). The
representation he chose was simply a list of 2D landmark locations. Each
location was part of a state matrix which was estimated at every iteration.
A list of landmark locations was chosen because it allowed the method to
have a low computational cost and use a small amount of memory, important
factors in the days of early computing. There have been many improvements
to the family of SLAM solutions which generate a list of point locations
since Smith's work. One of the first practical implementations on a real
robot was done by Thrun in 1998 \cite{Thrun1998}. In this work the SLAM
problem was posed in an Expectation Maximization (EM) framework which is
similar to the EKF framework in that landmark locations are saved
in a state vector which is estimated at every iteration. In Thrun's work an
occupancy grid map is generated as a post processing step from sonar
measurements. The results showed that their representation could become
more accurate over time by using new observations to improve the current
estimate. This a highly desired ability of any representation generation
method. The next step was the ability of these methods to include a loop
closure procedure. A loop closure procedure was proposed by Gutmann in 1999
\cite{Gutmann1999}. The key ability of the method was it could recognize
when the robot was revisiting a prior location and adjust the entire
representation with the constraint that the two points must coincide. In
2001 Dissanayake et al. \cite{Dissanayake2001} derived three theorems to
theoretically prove the convergence of the SLAM problem. Their test
platform used a millimeter-wave radar mounted on a vehicle and generated a
list of 2D landmark locations. In 2001 Thrun et al. \cite{Thrun2001} cast
the SLAM problem using particle filter techniques. Their results generated
a 2D map and showed an increased robustness and lower computational cost
than prior methods. One of the key disadvantages of methods up to this
point was that complexity scaled quadratically with the number of landmark
locations. In 2002 Montemerlo et al. \cite{Montemerlo2002} created a SLAM
solution named FastSLAM, which was able to handle a much larger number of
landmarks. They showed results with maps containing more than 50,000
points. Then, SLAM solutions using point locations became much more
directed towards 3D.

Some of the first interesting works that represented the world as a list of 3D
point locations were done by Thrun et al. in 2000 \cite{Thrun2000}, Liu and
Emery in 2001 \cite{Liu2001}, and H\"{a}hnel et al. in 2003. In these works the
2D landmark locations and robot position were estimated using techniques from
Thrun's past work \cite{Thrun1998}. Once this had been done the 3D laser scan
data was simply appended to each estimated robot location. Then, a mesh was
created by post processing the 3D point cloud. Their works utilized the fact that the
laser collected the data in an incremental manner and simply connected
neighboring 3D points. Finally, the mesh was simplified by looking for large
planar sections and merging the corresponding mesh elements. One of the first
SLAM solutions which used a single camera to generate a list of 3D points was
done by Davison in 2003 \cite{Davison2003}. Here he used a single camera to
generate a very sparse list of 3D points. This method was limited to small
environments. Future advances allowed representations of larger environments. In
2003 Thrun et al. \cite{Thrun2003} created a SLAM procedure which did not rely
on having a structured environment and was applied to mapping large mines. In
2004 Howard et al. \cite{Howard2004} created a SLAM system based on a Segway
platform equipped with a 3D laser which could map areas of roughly 0.5 km
on each side. One of the results showed a map with approximately 8 million
points. In 2006 Cole and Newman \cite{Cole2006} continued work in large-scale
SLAM by increasing robustness and also generated maps with many 3D points using
a laser sensor. In 2007 Clemente et al. created a large-scale SLAM system that
used a single camera. The system had an advanced loop closing procedure based on
visual features and created large maps of 3D points. In 2001 Klein and Murray
\cite{Klein2007} developed a SLAM solution which used a single camera. The
uniqueness of their method was the algorithmic structure. Their SLAM solution
consisted of two separate processes: a tracking process and a map building
process. This algorithmic structure has become very common in many current SLAM
solutions because of the advances in pose estimation technology. Klein and
Murray were able to get very good results for a small environment and showed
Augmented Reality (AR) applications. Many of the future advances of SLAM
solutions, which generated 3D point sets, dealt with camera systems
\cite{Paz2008,Konolige2008,Strasdat2010} and improved speed and robustness.
Many of the most current methods that produce a list of points are systems who
use a relatively new type of sensor named a RGB-D sensor. One good example is a
work that was produced in 2011 by Engelhard et al. \cite{Engelhard2011}. In this
work they used an algorithm named the Iterative Closest Point (ICP)
\cite{Rusinkiewicz} to align point clouds coming from the RGB-D sensor into
a large colored point cloud. The resulting maps were visually impressive.
However, the map could not be adapted to new information and was not well suited
for other applications, such as obstacle avoidance. These limitations are
inherent in maps that consist of lists of points.

\subsection{Volumetric}

Many SLAM solutions generate a 2D volumetric representation of the world
because they are especially advantageous in dealing with noisy sensors. Two
of the first major works that generated a 2D volumetric representation
were done in 1998 by Yamauchi et al. \cite{Yamauchi1998} and Schultz et al.
\cite{Schultz1998}. These works generated a 2D occupancy grid, which is a
type of volumetric representation. Here the environment was divided into a
2D grid. Each square of the grid contained the probability that it was
occupied with an object. All squares would be updated iteratively based on
the current sensor readings. Occupancy grids, like any other
volumetric-based representation, are limited by the amount of available
memory. In 2002 Biswas et al. \cite{Biswas2002} extended occupancy grid
methods by allowing dynamic environments. This was done by looking at past
``snapshots'' of the map. In 2004 Eliazar and Parr \cite{Eliazar2004}
continued the advancement by decreasing computational cost and implemented
a loop closure method.

There have been a few impressive SLAM solutions that generate a 3D volumetric
representation. There are three major works that generated a result very similar
to a 3D occupancy grid, which was saved as in a octree data structure
\cite{Magnusson2007,Nuchter2007,Huang2011,Endres2012}. Each work had a slightly
different name and procedure for generating the representation, but in general
the representations divided the environment into cubes and had a scalar value
representing the belief of a surface being there. Octrees were used to save
memory by only having a fine resolution of cubes at places where there was a
surface. There are many advantages to a 3D occupancy grid representation. The
representation is well suited for obstacle avoidance and path planning
applications. Also, the representation is very adaptable to new information. The
major disadvantage is that the representation can not be visualized immediately.
In order to render, an image must be generated at each desired viewpoint by ray
tracing the volume. This can be a problem when using such method for
applications such as teleoperation due to the computational cost of rendering.
The current state of the art for generating a volumetric representation was done
by Newcombe et al. in 2011 \cite{Newcombe2011a}. Their system used a RGB-D
sensor and generated a 3D voxelized grid Truncated Signed Distance Function
(TSDF) of the environment. For this type of representation each cube contains
the value of the distance to the nearest surface. The sign of the value is based
on which side of the surface the cube is relative to the sensor. This work has
been the most capable at dealing with extremely noisy data and dynamic scenes.
However, due to memory constraints the method can only represent environments
that are about the size of a 4m cube. Also, it must be ray traced in order to be
visualized.

\subsection{Surface}

One of the first major works that created a surface representation of the
environment in real-time was done by Martin and Thrun in 2002 \cite{Martin2002}.
Their method utilized an EM framework to fit plane models to 3D point cloud
data. Polygon mesh elements were then easily assigned to each plane. The main
drive behind this work was to generate a map of the environment that uses a
small amount of memory. Their  method worked well for structured environments.
One of the major limitations of their method, and other methods that only mesh
large planar sections, is that the representation will only consist of planar
sections and not capture the fine detail of the environment. In 2004 Viejo and
Cazorla \cite{Viejo2004} developed a methodology for generating a mesh that can
contain more information of the environment than large planar sections. Due to
this ability, they termed their method to be ``unconstrained.'' Essentially
their method was based on a 3D Delaunay triangulation algorithm. Giesen surveyed
Delaunay triangulation methods in \cite{Giesen2004}. Viejo and Cazorla were not
able to obtain real-time results and, in fact, it has been seen that it is
extremely difficult to run a 3D Delaunay triangulation in real time because of
the numerous distance calculations required. One of the next major advances
came from Weingarten and Siegwart in 2006 \cite{Weingarten2006}. Their work also
created a mesh that was only capable of capturing large planar surfaces.
However, they showed increased robustness. In 2007 Pollefeys et al.
\cite{Akbarzadeh2006,Pollefeys2007} developed a large urban mapping system
consisting of a vehicle and eight camera systems. The processing was carried out
by multiple CPUs and optimized for speed with Graphics Processing Unit (GPU)
calculations. In their work they used the camera systems to generate a set of
depth maps. This set was then reduced using their depth map fusion method. The
method combined multiple depth maps to reject erroneous depth estimates and
remove redundancy from the data, resulting in a more accurate and reduced set of
depth maps. The reduced set was then used by a triangulation procedure to create
a mesh of the environment. The mesh generation procedure was based on a work
from 2002 by Pajarola et al. \cite{Pajarola2002}. This method defines a mesh in
the depth image. It starts from a very coarse mesh and continues to refine in
areas of the depth image based on a confidence criteria. In the work of
Weingarten and Siegwart, these meshes that are defined for each fused depth
image are then checked for overlaps and duplicates are removed to make a single
large mesh. One of the major drawbacks of this approach is that the output mesh
can not be adapted by measurements that come from revisited parts of the scene.
Another major advancement came in 2008 from Poppinga et al. \cite{Poppinga2008}.
In this work they used a Time of Flight (ToF) camera to generate a mesh
representation of the large planar structures in the environment. Here they also
develop a procedure to determine a mesh in a depth image. They leverage the
structure of the depth image to make the method computationally inexpensive. In
their work they simply append the meshes that are created from each depth image
into a global coordinate system. They obtain very good results from a simple
method. However, the method is not adaptive to new information. Also, a mesh is
created for each depth image instead of updating and maintaining a global mesh.
A major advancement came from work done by Newcombe and Davison in 2010
\cite{Newcombe2010}. In this work they designed a method to create a mesh
reconstruction from a single video camera. Their method used Structure From
Motion (SFM) to obtain a sparse point cloud of the scene. Then an implicit
function was fit to the point cloud using the methodology of Ohtake et al.
\cite{Ohtake2003}. A bundle of depth maps is then selected. From the bundle a
single reference depth image is selected and a ``base'' model is constructed by
sampling the implicit surface for vertices in the reference frame. The
neighboring frames are used to better the ``base'' model and create a more
accurate mesh. Each reference frame has its own mesh and all the meshes are put
into a global coordinate system. Duplications are then detected and removed.
Again, the representation is not adaptive to new information. In 2010
St\"{u}hmer et al. \cite{Stuhmer2010} generated very accurate depth maps from
several color images in real-time. They showed very impressive results but their
method was not designed to maintain a representation in a global coordinate
frame.

The next major advances in methods that generated surface representations of the
environment, were based on RGB-D sensors. This type of sensor has become very
popular since the release of the Kinect from Microsoft that was the first mass
produced RGB-D sensor of its kind. RGB-D sensors are inexpensive and produce
noisy 640x480 depth images at 30Hz. The RGB-D sensor has excited the robotics
community because this has been the first time that depth data has been so
readily accessible from such an inexpensive sensor. Therefore, these
methodologies must be able to quickly deal with very high rates of information.
One impressive work came from Henry et al. in 2012 \cite{Henry2012}. In this
work they designed a system that used a RGB-D sensor to build a map made of
surfels (Surfels are circular disks which have a particular position and
orientation and also a radial size based on confidence.). In order to generate
and maintain the surfel map they used the work of Weise et al. \cite{Weise2009}.
The map consists of a large number of surfels. The surfel map can be updated
given new registered depth images from the sensor. Decisions are made how to
handle each measurement in the depth image based on the difference between an
expectation generated using the current map and the actual readings from the
sensor. Rendering a surfel map requires special methods \cite{Pfister2000} and
is difficult to use in applications such as obstacle avoidance.

One of the next major advances is a highly-related work that was published by
Whelan et al. in 2012 \cite{Whelan2012} and more recently in 2013
\cite{Whelan12tr}. The system they developed was named Kintinuous and was able
to produce a high quality mesh representation of the environment. Their hybrid
system utilized the KinectFusion method \cite{Newcombe2011a} of Newcombe et al.
to create a volumetric representation of the portion of the environment in front
of the sensor. As the sensor moves, portions of the environment that leave the
volume in front of the sensor are ray cast and turned into a mesh. They obtain
very impressive results but also mention a limitation of their system for future
work. The limitation is that the mesh can not be updated once created, which is
an issue when revisiting parts of the environment that may have changed. One of
the most impressive current works which has an adaptable mesh came from Cashier
et al. in 2012 \cite{Cahier2012}. In this work, they were able to generate and
update a mesh with new measurements from a ToF sensor. They used the difference
between the existing model and the actual measurements to decide whether to
adapt the mesh or add new elements. The mesh topology was not adaptive to the
environment and their experiments only showed results of mapping a single flat
wall with no robot movement. The system needs to be tested for object addition
and removal.

\section{Surface Reconstruction}

The computer graphics field has spent considerable effort to develop
methodologies for creating representations from sets of data. Generally, these
sets of data are acquired from a sensor. Methodologies have progressed steadily
and are often designed for a specific application. One of the original
motivations was to generate surfaces from medical imaging data. This improves a
doctor's  decisions because the data are presented in a more intuitive manner.
Current applications include augmented reality and 3D printing. Older
methodologies were not as concerned with speed and often times had a large
computational cost. Also, the methodologies are often designed for single
objects or small environments. Following the taxonomy of such well known works
as \cite{Gopi2002,Mencl1997}, the field can be roughly divided into
representations that are generated with volume-based techniques and those that
use surface-based techniques. Methods that use volume-based techniques are
characterized by spatially subdividing the environmental volume and are usually
computationally expensive and require a large amount of memory. Methods that use
surface-based techniques generate the representation using surface properties of
the input data. Both types of methods can have mechanisms to adapt the mesh to
noisy or new information. In the following section we will trace the progression
of the methodologies.

\subsection{Volume-based}

Volume-based methods have the characteristic of spatially subdividing the volume
into smaller parts. One of the first well known works that used a volume-based
technique was proposed by Lorensen and Cline in 1987 \cite{Lorensen1987}. In
this work they proposed a method named marching cubes, which is still known for
its reliability and simplicity and is used by applications that do not have a
computational requirement. Marching cubes subdivides the space into cubes. The
data contained in each cube dictate how the surface connectivity will be defined
in that cube. Possible vertex locations are at the corners and along the edges.
Once this has been done for all cubes the process is complete. One of the next
major steps came from Hoppe et al. in 1992 \cite{Hoppe1992} In this work they
used the input points to define a Signed Distance Function (SDF) in 3D space and
then meshed the zero-set to obtain the output mesh. A SDF is a spatial function
that has the value of the distance to the nearest surface at each point. The
sign is used to specify if the point is inside or outside of the surface
relative to the sensor. The zero-set of the SDF is the surface where the values
transition from positive to negative. Using a SDF has proven to be very
effective and has been the core idea of many methodologies that came after this
work of Hoppe et al., such as KinectFusion \cite{Newcombe2011a}. One of the next
advances came from Edelsbrunner and M\"{u}cke in 1994 \cite{Edelsbrunner1994}
with a method named alpha shapes. They used 3D Delaunay triangulation and the
input point set to decompose the volume into a Delaunay tetrahedrization. This
gives a triangulation of the input set which involves all points. A sphere of
radius alpha is then used to remove edges and vertices to obtain a mesh of user
specified resolution. Many works have made use of 3D Delaunay triangulation to
create a mesh. Methods which use 3D Delaunay on the input set have a large
computational cost and often cannot be executed in real-time. The next valuable
contribution came from Bloomenthal in 1994 \cite{Bloomenthal1994} as open source
software for surface polygonization of implicit functions. This was a stable and
robust open source software that has been used in many well-known algorithms
\cite{Newcombe2010}. Another major advance came from Curless and Levoy in 1996
\cite{Curless1996}. In this work they also constructed a Truncated Signed
Distance Function (TSDF). A TSDF is very similar to a SDF; the only difference
is that distance values are truncated after they exceed a threshold. Their
method was one of the first to be able to handle several registered range scans.
Their work showed how well a TSDF can deal with several noisy scans by naturally
integrating out the noise. They obtained very good results but were not even
close to real-time. A speed up in processing time was achieved by Pulli et al.
in 1997 \cite{Pulli1997} by utilizing octrees. They obtained good results and
their method was used by Surmann et al. \cite{Surmann2003} in a well-known
robotic mapping work. Another major advance came in 2001 from Zhao et al
\cite{Zhao2001}. They used Partial Differential Equation (PDE) methods to obtain
a final reconstruction that was of better quality than prior methods. In 2001
Carr et al. \cite{Carr2001} created a volumetric method based on the radial
basis function (RBF). Their method was able to successfully deal with holes and
generate water tight models. A water tight model is useful for single object
reconstruction. However, it is not desired for mapping large environments. One
of the next major advances was published in 2003 by Ohtake et al.
\cite{Ohtake2003}. In this work they created a method that was faster than the
work of Carr et al. \cite{Carr2001} by implementing a hierarchical approach with
compactly supported basis functions. At the time, their work was considered to
be the state of the art for calculating an implicit function of a noisy point
set and was used by Newcombe et al. \cite{Newcombe2010}. Volume-based methods
have been able to create high quality representations and work well for single
objects and small environments. These methods must spatially divide the
environmental volume and therefore have a high memory requirement.

\subsection{Surface-based}

One of the first interesting and adaptive surface-based methods was published by
Terzopoulos and Vasilescu in 1991 \cite{Terzopoulos1991a} and dealt with 2.5D
data such as intensity and range images. The goal of their work was to create an
adaptive mesh of an input image. The mesh was initialized as a 2D sheet of mesh
elements with virtual springs along each edge. The stiffness of each virtual
spring would then adjust based on the image information at its locations. The
mesh was able to adapt to be more dense in regions of higher intensity. In 1992
Terzopoulos and Vasilescu extended their methodology to 3D data
\cite{Vasilescu1992}. In this work they used the distance between the mesh and
the data to drive the vertices to be near the surface. In this early work they
needed to initialize the mesh and control the subdivision of mesh elements to
obtain a suitable resolution. In 1993 Hoppe et al. \cite{Hoppe1993} published a
method that used an energy minimization framework. Their method minimized an
energy function that modeled the competing desires of conciseness of
representation and fidelity to the data. They successfully used their method for
both surface reconstruction and mesh simplification. One of the next advances
in physical based adaptation of meshes came in 1993 from Huang and Goldof
\cite{Huang1993}. In this work they were able to adjust the size of the mesh
elements to obtain a denser resolution in areas of high frequency information
using a physical based model. In addition, it was one of the first works to
represent an object undergoing deformation. Their method was able to perform
tracking on simple simulation examples. Another advancement came in 1994
Rutishauser et al. \cite{Rutishauser1994} with a method specifically designed
for incremental data. Their methodology worked with a sequential input set of
range data and used a probabilistic framework to adjust the vertices of a mesh
to the expected value given the prior observations. Their methodology also
modeled the noise of the sensor with a sensor model. In 1994 Delingette
\cite{Delingette1994} developed a methodology to generate a simplex mesh model
of structured and unstructured 3D datasets. Elastic behavior of the mesh surface
was modeled by local stabilizing functionals. Also, they implemented an
iterative refinement process to refine the mesh in areas of high frequency
information. One of the next steps was published by Turk and Levoy in 1994
\cite{Turk1994}. Their method allowed overlapping meshes to be ``zippered'' into
a single mesh surface. This ability is especially important for methods that
generate a mesh for each depth image of the sensor and then need to combine all
registered meshes into a single mesh. Their method is computationally expensive
due to distance calculations. An interesting work came in 1995 from Chen and
Medioni \cite{Chen1995}. They devised an adaptive mesh methodology based on the
inflation of a balloon. A mesh sphere was first initialized within the
registered range measurements of the object. Virtual inflation forces were then
used to expand the balloon until the mesh surface was a minimal distance from
the range data. This method was limited to objects that are water tight. A
major advancement came in 1999 from Bernardini et al., \cite{Bernardini1999a} in
a method named the ball-pivoting algorithm. Their method is a good example of an
advancing front method. These types of algorithms start with a seed mesh element
and advance the boundary by adding new mesh elements in the immediate area of
the boundary which is supported by measurements. Advancing front algorithms
differ in how it is decided to add new mesh elements. In the work of Bernardini
et al., a virtual sphere of a user defined radius is rolled along the boundary of
the mesh and new elements are added if the ball touches another measurement.
Their methodology became popular because of its simplicity. One major
disadvantage was that the generated mesh was a fixed topology. Another advancing
front method came in 2001 from Gopi et al. \cite{Gopi2001,Gopi2002}. Here, they
sampled the input dataset to obtain a new dataset with a lower density of points
in areas of lower frequency information. This effectively gave their method an
adaptive topology. Next, a local neighborhood was computed at each data point
and projected to a plane tangent to the surface. The triangulation is then
computed on this local tangent plane. They obtained impressive results on
datasets of varying sample density and curvature. An interesting work was
published in 2003 by Ivrissimtzis et al. \cite{Ivrissimtzis2003}. Here they used
a neural network model to adapt a mesh model to the data. They claimed that
their method is computationally independent of the size of the input dataset
because the dataset is only sampled by the method. There obtained good results.
In 2004 Alexa et al. published a very interesting work to generate point set
surfaces from an input dataset \cite{Alexa2004}. They use moving least squares
(MLS) to locally approximate the surface with polynomials. The original dataset
is then no longer used. Instead, they develop tools to sample the approximated
surface to any resolution desired so that the end result is another point set of
user specified resolution lying closer to the surface than the input dataset.
One drawback is they had to develop their own methodology to render a point set.
In 2005 Scheidegger et al. used the work of Alexa et al. to develop an advancing
front methodology to generate concise meshes of high accuracy. Their main
contribution was to augment an advancing front algorithm with global information
so that the triangle size could adapt gracefully to any change. They obtained
very impressive results. Most methodologies in Surface Reconstruction had been
solely concerned with object or small environment recreation and have
computational or memory requirements that do not work well with large
environments. One of the first successful methods intended for large
environments was published in 2009 by Marton et al. \cite{Marton2009}. Their
methodology was an advancing front algorithm that worked on a point set sampled from the MLS surface of the original point set. They were able to
obtain impressive and near real-time results on datasets of large environments.
They also developed a method to deal with revisited parts of the scene by
determining the overlapping area and reconstructing only the updated part of the
surface mesh. To support dynamic scenes they developed mechanisms to decouple
and reconstruct the mesh quickly. They only discussed these mechanisms in theory
and had no results of how these mechanisms work.

\section{Summary}

The fields of Robotics and Computer Vision have developed many exciting
methodologies to construct representations from a noisy input dataset. However,
there is still work to be done to obtain the ideal reconstruction method. A mesh
is clearly a desirable type of representation. An ideal method both
generates and maintains a mesh representation efficiently. Also, many existing
methods do not leverage the inherent structural information contained within the
depth image. There are imaging processing techniques that could be used to
answer some of the remaining problems in surface reconstruction, such as the
need for adaptive topology and the need to decide how each measurement should be
used to update the existing mesh. Henry et al. \cite{Henry2012} have already
investigated using the difference between the expected and actual measurements
to guide the decision of how to use each measurement. However, their work was
intended for surfels and needs to be extended to meshes. A method to generate a
representation is needed which is computationally and memory efficient and can
adapt the representation to new information.
