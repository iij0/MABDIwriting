\documentclass[11pt,titlepage]{article}
%::::::::::::::::::::::::::::::::::::::::::::::::::::::::::::::::::::::::
%                      AMS-LaTeX packages
%  These packages include the functions of AMS-LaTeX.
%::::::::::::::::::::::::::::::::::::::::::::::::::::::::::::::::::::::::
\usepackage{amsmath}
\usepackage{amsbsy}
\usepackage{amsopn}
\usepackage{amsthm}
\usepackage{upref}
%::::::::::::::::::::::::::::::::::::::::::::::::::::::::::::::::::::::::
%                      AMSFonts packages
%  These packages give access to the various AMS fonts.
%::::::::::::::::::::::::::::::::::::::::::::::::::::::::::::::::::::::::
\usepackage{amsfonts}
\usepackage[mathscr]{eucal}
\usepackage{eucurs}
%::::::::::::::::::::::::::::::::::::::::::::::::::::::::::::::::::::::::
%                      Graphics packages
%  These packages allow PostScript figures to be included.
%::::::::::::::::::::::::::::::::::::::::::::::::::::::::::::::::::::::::
\usepackage{epsfig}
\usepackage{psfrag}
%::::::::::::::::::::::::::::::::::::::::::::::::::::::::::::::::::::::::
%                       Table and Figure packages
%  These packages modify the 'tabular' and 'array' environments.  Also 
%  they modify the 'caption' command and create a 'subfigure' environment.
%::::::::::::::::::::::::::::::::::::::::::::::::::::::::::::::::::::::::
\usepackage{array}
%\usepackage{tabularx}
\usepackage{hhline}
\usepackage[sl,bf,hang,small]{caption}
%::::::::::::::::::::::::::::::::::::::::::::::::::::::::::::::::::::::::
%                          List packages
%  This package modifies the 'enumerate' environment.
%::::::::::::::::::::::::::::::::::::::::::::::::::::::::::::::::::::::::
\usepackage{enumerate}
%::::::::::::::::::::::::::::::::::::::::::::::::::::::::::::::::::::::::
%                      Miscellaneous packages
%::::::::::::::::::::::::::::::::::::::::::::::::::::::::::::::::::::::::
\usepackage{fancyheadings}
\usepackage{fancybox}

\newlength{\enumerateindent}
\setlength{\enumerateindent}{\leftmargini}
\addtolength{\enumerateindent}{-\labelsep}
\addtolength{\enumerateindent}{-0.4\labelwidth}

\setlength{\textheight}{8.75in}
\setlength{\textwidth}{6.9in}
\setlength{\oddsidemargin}{-0.15in}
\setlength{\evensidemargin}{-0.15in}
\setlength{\topmargin}{-0.4in}
\setlength{\headheight}{5ex}
\setlength{\footskip}{5ex}
\setlength{\parskip}{2ex}
\setlength{\parindent}{1em}

\renewcommand{\floatpagefraction}{1.0}
\renewcommand{\textfraction}{0.0}
\renewcommand{\topfraction}{1.0}
\renewcommand{\bottomfraction}{1.0}
\setlength{\textfloatsep}{2ex}

%\renewcommand{\captionmargin}{50pt}

\newcommand{\bm}[1]{\boldsymbol{#1}}
\newcommand{\er}[1]{\mathcur{#1}}
\newcommand{\es}[1]{\mathscr{#1}}

\newcolumntype{C}{>{$}c<{$}}
\allowdisplaybreaks
%\renewcommand{\tabularxcolumn}[1]{m{#1}}

\def\AmS{{$\mathcal{A}$\kern-.14em\lower.5ex\hbox{$\mathcal{M}$}%
\kern-.05em$\mathcal{S}$}}
\def\AmSLaTeX{\protect\AmS-\protect\LaTeX}
\def\TiPs{T\kern-.1667em\lower.5ex\hbox{I}\kern-.1emP%
\kern-.09em\raise.48ex\hbox{\small S}}
\def\TeXTiPs{\TeX\kern-.1em\TiPs}
\def\PiC{P\kern-.12em\lower.5ex\hbox{I}\kern-.075emC}
\def\PiCTeX{\PiC\kern-.11em\TeX}
\def\BiB{B\kern-.12em\lower.5ex\hbox{I}\kern-.075emB}
\def\BiBTeX{\BiB\kern-.11em\TeX}
\def\AucTeX{{\scshape A\kern-.06emu\kern-.035emc}\kern-.06em\TeX}
\def\Xy-pic{\kern-.1em X\kern-.3em\lower.4ex\hbox{Y\kern-.15em}-pic}

\pagestyle{fancy}
\lhead{\Large\TeXTiPs \vspace{0.25ex} }
\rhead{Page \thepage \vspace{0.25ex} }
\cfoot{Last Revision: \today}

\begin{document}
%
\thisfancyput(3.3in,-4.5in){%
\setlength{\unitlength}{1in}%
\thicklines%
\fancyoval(6.125,8.75)%
\thinlines%
\fancyoval(6.025,8.65)}
\title{{\fontfamily{cminch}\selectfont \Huge{\TeXTiPs}\\[0.5ex]} Supplementary
  information about \LaTeXe \\
 \large{\underline{Second Edition}}}
%\title{{\fontfamily{cminch}\selectfont \TeXTiPs\hspace{6pt}}%
%{\Huge $\stackrel{\textstyle\bullet\rule[-15pt]{0pt}{25pt}}{\bullet}$}%
%\\[2ex] Some suggestions for using \LaTeXe}
\author{James W. Howse, Los Alamos National Laboratory, \\
        Neall E. Doren, Sandia National Laboratories}
\maketitle
%
\noindent
This article is a repository of helpful advice concerning the use of \TeX\ and
some of the more common macro languages for it, specifically \LaTeXe\ and
\AmSLaTeX .  Currently these pages contain the knowledge of only one
person\footnote{specifically, James Howse}, but it is to be hoped that others
will add to this as time goes on.  For those of you using \TeX\ for the first
time, there are several things which are important to keep in mind.  First of
all, it is very important to understand that \TeX\ is \underline{NOT} a word
processor.  It is nothing like $\text{\textsc{Word}}^{\circledR}$ or
$\text{\textsc{WordPerfect}}^{\circledR}$ in that it has no built-in text
editor, page previewer, spelling checker, or printer drivers.  \TeX\ was
written in the dark ages of computing before these additions were believed to
be of any use.  It is much more helpful to think of the commands in \TeX\ as
representing a programming language in which the user describes the desired
document layout in a procedural manner.  The program \texttt{tex} compiles
this language into a machine \textbf{\textit{independent}} binary file which
is denoted by the extension \texttt{.dvi}.  \LaTeXe\ and \AmSLaTeX\ are macro
packages for \TeX\ whose commands are written in the \TeX\ language.  Second,
as one sage was heard to say, ``\LaTeX\ is just a hack''\footnote{attributed
  to Bill Horne}.  For the most part, no one wants to program in \TeX, so
higher level languages, such as \LaTeXe\ and \AmSLaTeX, were developed from it
that are hopefully easier to learn and use.  The difficulty for the end user
is that some of the commands in these languages implicitly contain very
specific concepts of how a page of text should be laid out.  This presents no
problem so long as the user's and the program author's concepts of proper page
layout agree.  However, if they do not, it is in some cases extremely
difficult to obtain a specific page format.

\noindent
\shadowbox{\parbox{17.1cm}{
{\bf NOTE:}  The documentation files described herein can be downloaded via
the web from from the site 
\textsf{http://www.math.unm.edu/\hspace{.9mm}$\bf\tilde{}$\hspace{.3mm}nedoren/latex/textips\_docs}\;.  However, they may already be present on your local
system.  Ask your administrator for their location, or for
help in installing these utilities and documents,  if required.  }}

As was previously mentioned, \TeX\ does not have a built-in text editor, page
previewer, spell checker, or output device drivers.  However, there are
programs available which perform each of these complementary functions.  A
list of public domain programs for this purpose is documented in
Table~\ref{tbl:aux_prog_tex}.
%
\begin{table}[htbp]
\setlength{\topsep}{0in}
\begin{center}
\begin{tabular}{|>{\centering}m{0.85in}||m{5.5in}|}
\hhline{--}
\multicolumn{2}{|c|}{\rule[-0.075in]{0in}{0.25in}\Large\textbf{Programs to
  complement \TeX}}
\\ \hhline{|=:t:=|}
\textsl{Program} & \multicolumn{1}{c|}{\textsl{Purpose of the program}} 
\\ \hhline{-||-}
\texttt{dvips} & Translates the compiled output of \texttt{tex} (i.e., the
binary \texttt{.dvi} file) into the \textsc{PostScript} page description
language (i.e., the ASCII \texttt{.ps} file).  This program has both a man
page and external documentation, which is in the file
\verb+dvips-5.58.dvi+. 
\\ \hhline{-||-}
{\texttt{xdvi \\ xtex}} & 
Both are X-Windows screen previewers which provide an almost WSYWIG display of
a compiled \texttt{tex} output file (i.e., the binary \texttt{.dvi} file).
The only documentation for either of these programs is obtained by using the
command \texttt{man} \textit{program}.  These man pages can be printed
using the command \texttt{man -t} \textit{program}.  Note that
\texttt{xtex} has \textbf{\textit{not}} been kept up to date by its authors.
\\ \hhline{-||-}
\texttt{ghostview} & An X-Windows screen previewer which provides an almost
WSYWIG display of a \textsc{PostScript} file (i.e., the ASCII \texttt{.ps}
file).  The only documentation for this program is obtained by using the
command \texttt{man ghostview}.  These man pages can be printed using the
command \texttt{man -t ghostview}. 
\\ \hhline{-||-}
\texttt{ps2pdf} \\ $\text{\textsc{Distiller}}^{\textsc{tm}}$ \\ \texttt{pstopdf}
& Programs for converting \textsc{PostScript} generated by \texttt{dvips}
into  $\text{\textsc{Adobe}}^{\circledR}$ \texttt{PDF} format.
The Aladdin \texttt{ps2pdf} and Derek Noonburg's \texttt{pstopdf} (preferred)
programs are
freeware converters for both the 
$\text{\textsf{Unix}}^{\circledR}$
and  $\text{\textsc{MS-Windows}}^{\textsc{tm}}$ platforms.
$\text{\textsc{Distiller}}^{\textsc{tm}}$ is available  from
$\text{\textsc{Adobe}}^{\circledR}$ for 
$\text{\textsc{MS-Windows}}^{\textsc{tm}}$ (at a significant cost).  
It has been observed that $\text{\textsc{Distiller}}^{\textsc{tm}}$
produces a tighter (smaller) output file, with no loss of output quality.
Find \texttt{pstopdf} at \verb+http://www.foolabs.com/xpdf/+ .
\\ \hhline{-||-}
\texttt{ispell} & A spelling checker which is able to filter out most \TeX\
and \LaTeXe\ commands.  This program also has non-English dictionaries and
allows the user to build a personal dictionary. This program has both a man
page and external documentation, which is in the file
\verb+ispell-3.1.dvi+.
\\ \hhline{-||-}
\texttt{bibcard} & An X-Windows utility for simplifying the often daunting
task of submitting bibliographic entries to the \BiBTeX database.  
Pull-down menus for book, article, thesis, conference and technical report
citations are at the touch of the mouse.  The source code for this freeware
program compiles easily on most workstations and is available from 
\textsf{ftp.iam.unibe.ch} in directory \texttt{/pub/X11/Bibcard-1.0.tar.Z}.
\\ \hhline{-||-}
\texttt{emacs} & A text editor that has a number of different editing modes,
each well suited to a particular task.  The editor has an extensive array of
features, all of which can be customized by the user. Arguably the best text
editor available on a $\text{\textsf{Unix}}^{\circledR}$ platform.  This
program has both a man page and an \textbf{\textit{extensive}} interactive
help facility. The external documentation is in the file
\verb+emacs-19.22.dvi+.
\\ \hhline{-||-}
\AucTeX & This is not a program, but rather a macro package for
\texttt{emacs}. It defines a new \TeX/\LaTeXe\ mode for \texttt{emacs} that 
allows all of the above programs to be used from within a single integrated
environment. Furthermore, it defines power keys which expedite the entry of
many \TeX\ and \LaTeXe\ commands.  Those of you interested in this package
will probably have to install it yourself.  Documentation and installation
instructions are in \verb+auctex-9.3.dvi+.  The ftp site is
\textsf{sunsite.auc.dk} in \texttt{/packages/auctex/}.
\\ \hhline{--} 
\end{tabular}
\end{center}
\caption{Auxiliary programs which complement \TeX.}
\label{tbl:aux_prog_tex}
\end{table}
%
In addition to document formatting, \texttt{emacs} has editing modes for
handling e-mail, writing programs, and a huge number of other things.
Although \texttt{xdvi} and \texttt{xtex} are both \texttt{.dvi} file
previewers, they have somewhat different features.  \texttt{xdvi} handles
\textsc{PostScript} far better than \texttt{xtex}, particularly if
\textsc{PostScript} fonts are used in the main body of the document.  Also the
screen fonts of \texttt{xdvi} have a nicer appearance than those of
\texttt{xtex}.  On the other hand, \texttt{xtex} has some useful features that
\texttt{xdvi} lacks.  For example, \texttt{xtex} has a mouse driven ruler,
which is very helpful if your document must conform to some type of page
specification.  This is common practice both for articles in conference
proceedings and for masters theses and doctoral dissertations.  Furthermore,
\texttt{xtex} has commands that can be embedded in your document which display
a red box around all cross references in the document.  Pressing the mouse
button within any one of these boxes takes you to the occurrence of that cross
reference.  For instance, at a bibliographic citation, you can press the mouse
button in the red box around the reference number and you are taken to the
point in the bibliography where the actual citation appears.  Releasing the
mouse button returns you to your original position in the document.  Also
\texttt{xtex} keeps track of the page number as part of its interface, which
can be very helpful if the document has no page numbers, or it is inconvenient
to be constantly moving to the part of the page on which they are located.
Note that neither of these programs displays documents containing color
properly.  In order to preview such documents, convert the \texttt{.dvi} file
to a \textsc{PostScript} \texttt{.ps} file by executing the command
\texttt{dvips} \textit{file} \texttt{-o}, then display the resulting
\texttt{.ps} file with \texttt{ghostview}.  By default, the output of the
\textsc{PostScript} translator \texttt{dvips} is sent to the default printer
for the computer.  The printer that the output is sent to, can be changed for
the duration of a login session, by executing the command 
\verb+setenv PRINTER+ \textit{printer} \textbf{\textit{before}} running
\texttt{dvips}.  To alter the printer used for only one print job, use the
command \verb+dvips -P+\textit{printer} \textit{file}.

Earlier, the \TeX\ macro packages \LaTeXe\ and \AmSLaTeX\ were mentioned.  The
commands available in \LaTeXe\ are discussed in \cite{Lam94a}.  It can be
purchased for about \$36, or try to borrow a copy of it from the system
administrators, a professor, or a fellow student.  \AmSLaTeX\ is a macro
package, written under commission for the American Mathematical Society, whose
commands are a superset of \LaTeXe 's.  Some of the additional features of
\AmSLaTeX\ are
%
\vspace{-\parskip}
\begin{enumerate}[\hspace{\enumerateindent}1) ]
\item Seven new alignment formats for equations.  For example, these alignment
  formats allow the following two constructions which would be rather
  difficult in \LaTeXe.  The first construction is
\begin{gather}
\hspace*{-10em}
\begin{align*} \varphi(x,z)
&=z-\gamma_{10}x-\sum_{m+n\ge 2}\gamma_{mn}x^mz^n\\
&=z-Mr^{-1}x-\int_{2}^{\rule{0.08em}{0in}\infty} Mr^{-\tau}x^{(\tau-n)}z^{(\tau-m)} \, d\tau
%\rule[-1ex]{1pt}{1ex}
\end{align*}\\[6pt]
\hspace*{-10em} \zeta^0=(\xi^0)^2,\\
\hspace*{-10em} \zeta^1=\xi^0\xi^1 + \xi^1\xi^2 + \xi^0\xi^2,\\
\hspace*{-10em} \zeta^2=(\xi^1)^2 + \xi^0\xi^1.
\end{gather}
The second construction is
\begin{alignat}{2}
x_0& = y_0 && \qquad \text {by (A.3)} \\
x_1& = y_1 + 5 && \qquad \text {by (A.11)} \\
x_0 + x_1& = y_0 + y_1 + 1 && \qquad \text {by Axiom 1.}
\end{alignat}
\item Two new math environments, \texttt{cases} and \texttt{matrix}.  The
  \texttt{cases} environment allows the construction of equations such as
\begin{equation} P_{r-j}=
  \begin{cases}
    0&  \text{if $r-j$ is odd},\\
    r!\,(-1)^{\frac{r-j}{2}}&  \text{if $r-j$ is even}.
  \end{cases}
\end{equation}
The \texttt{matrix} environment greatly reduces the number of commands that
must be entered in order to create a matrix.
\item A \texttt{proof} environment and three different theorem styles;
  \texttt{theorem}, \texttt{definition} and \texttt{remark}.
\end{enumerate}
\vspace{-\parskip}
%
Complete documentation of all the features of \AmSLaTeX\ is in the user's
manual which is in the file \verb+amsldoc.dvi+.

The problem with both \LaTeXe\ and \AmSLaTeX\ is that some of the commands
make implicit assumptions about the layout of the text.  If the user wants the
text to appear differently, this can sometimes be a {\Large\emph{BIG}}
problem.  Also there are functions that are not in either \LaTeXe\ or
\AmSLaTeX\ which would, on various occasions, be extremely useful.  One
solution to these problems is for the user to try to coerce \LaTeXe\ or
\AmSLaTeX\ into doing what he wants.  This works very well for certain
problems, and not at all for others.  Another solution is to revise or extend
the existing commands.  One way of doing this is to write your own commands.
While this is a potentially interesting adventure, in many cases it is likely
to be quite time consuming, as most commands to accomplish anything of moment
must be written in \TeX.  Another approach which is usually less time
consuming is to use revisions or extensions that others have written.  There
are a large number of these available, so before writing your own functions,
it may be helpful to see whether one already exists.  The first thing to check
if you have a problem that is not resolved by \cite{Lam94a} is the \TeX -FAQ
(i.e., Frequently Asked Questions).  This is available via the World Wide Web
at the site
\verb+http://www.cogs.susx.ac.uk/cgi-bin/texfaq2html?introduction=yes+.  It is
also posted monthly to two bulletin boards, \textsf{comp.text.tex} and
\textsf{news.answers}.  It can also be obtained by anonymous ftp from the
server \textsf{rtfm.mit.edu} in the directory
\texttt{/pub/usenet/news.answers/tex-faq}.  The next place to look for
solutions to your problems is \cite{Goo94a}.  This book discusses many of the
packages that extend or modify the basic capacities of \LaTeXe.  This book is
excellent because it is laid out based on the user's desired outcome, rather
than being a reference manual.  For instance there is an entire chapter
describing how to make floating bodies behave in specific ways.

%::::::::
% Beware \linebreaks
%::::::::
The revisions and extensions of \LaTeXe\ are called packages.  Note that
\AmSLaTeX\ is merely a set of packages for \LaTeXe.  Packages are read into a
document using the command \verb+\usepackage[+\textit{package \linebreak
  options}\verb+]{+\textit{package name}\verb+}+.  These commands are usually
entered immediately after the command \linebreak
\verb+\documentclass[+\textit{class options}\verb+]{+\textit{class
    name}\verb+}+, which is on the first line of the document.  Generally
speaking, the \textit{class name} determines the overall format of the
document, and \textit{package name} determines specific aspects of the
document format.  The \textit{class options} and \textit{package options} are
flags which select between specific format choices in the class or package.
These options are a list of terms separated by commas and containing
\textbf{\textit{no}} blank spaces.  In \LaTeXe\ the extensions and revisions
that have been written by others are designed to be entered as packages.
These packages are usually denoted by files whose names end in \texttt{.sty}.
Packages whose file name ends in \texttt{.sty} can be loaded using the
\verb+\usepackage{+$\cdot$\verb+}+ command \textbf{\textit{without}} this
extension.  The following list describes some useful styles for \LaTeXe.  They
are invoked with the \verb+\usepackage{+\textit{package name}\verb+}+ command
unless otherwise indicated.
%
\vspace{-\parskip}
\begin{description}
\item[\mdseries\AmSLaTeX:] As stated previously \AmSLaTeX\ is a package
  written for the American Mathematical Society which greatly expands the
  mathematical capacities of \LaTeXe.  Most of the useful features are loaded
  using the command \verb+\usepackage{amsmath,amsbsy,amsopn,amsthm,upref}+.
  Complete documentation is contained in the file \verb+amsldoc.dvi+.
\item[\textmd{\texttt{endfloat.sty:}}] Many journals require that all tables
  and figures appear at the end of the document, each on a separate page.
  Further, a list of both tables and figures is often required.  This style
  file removes tables and figures from within a document, transfers them all
  to the end, indicates where they were in the original, puts them all on
  separate pages, and compiles separate lists of tables and figures.
  Note that this package should only be loaded \textit{\textbf{after}} your
  document has been finished with all of the figures in their correct
  locations.  Documentation is found in the file \verb+endfloat.dvi+.
\item[\textmd{\texttt{theapa.sty \& theapa.bst:}}] These two files define a
  bibliographic citation format that is used by the American Psychology
  Association (APA).  The biggest difference between this format and the
  standard \LaTeXe\ one is that rather than citing references by number from
  the bibliography, they are cited alphabetically by the first author's last
  name and the year of publication.  This citation format is required by some
  journals, for instance \textit{Neural Networks}. The first file defines the
  way citations appear in the main body of the paper, the second defines the
  way that they appear in the bibliography.  In order to use this format, the
  commands \verb+\usepackage{theapa}+ and \verb+\bibliographystyle{theapa}+
  must be placed into your document.  The documentation is in the file
  \verb+theapa.doc+.
\item[\textmd{\texttt{fancyheadings.sty:}}] This file defines a format to
  customize the headers and footers on each page.  Both the headers used in
  the \cite{Lam94a}, and the headers and footers in this document, are
  examples of the sort of customization that can be done.  The package is
  documented in the file \verb+fancyheadings.dvi+.
\item[\textmd{\texttt{fancybox.sty:}}] This file defines commands which allow
  a box to be drawn around virtually anything.  Equations, lists, floats, and
  entire pages are among the things that can be boxed.  The box on the title
  page of this document is an example of what can be done with this package.
  The documentation for the package is in the file
  \verb+fancybox.dvi+, and it contains many useful examples of how to
  box various things.
\item[\textmd{\texttt{merge.sty:}}] This file defines a format for producing
  form letters by merging addresses, openings and closings into a standard
  main body.  Note that the file containing the addresses, etc., can
  \textbf{\textit{not}} contain \textbf{\textit{any}} blank lines, even at the
  end of the file.  The package is documented in the file
  \verb+merge.doc+.
\item[\textmd{\texttt{setspace.sty:}}] This package provides commands to set
  arbitrary line spacing in documents.  Professors often request double
  spacing for drafts of papers, as it gives them more space in which to write
  encouraging remarks.  The package provides three commands,
  \verb+\doublespacing+, \verb+\onehalfspacing+, and \verb+\singlespacing+,
  for setting the overall spacing for the document in the preamble.  There are
  also three environments with the same names which allow the spacing to be
  changed within the document.  If a different spacing is required initially,
  then the \verb+\setstretch{+\textit{spacing multiplier}\verb+}+ command can
  be used in the preamble to set \texttt{baselinestretch} appropriately.  The
  package is documented in the file \verb+setspace.doc+.
\item[\textmd{\texttt{subequations, subfigure.sty:}}] The first list entry is
  an environment that is contained in \AmSLaTeX , and the second entry is a
  package.  These environments allow related equations or figures to be given
  the same reference number followed by different letters.  Note that the
  individual members of the subequation or subfigure can be referenced
  separately by using a different \verb+\label{+\textit{name}\verb+}+ command
  for each member.  The equation
  \begin{subequations}
    \begin{align}
      \zeta^0 & = (\xi^0)^2,\\
      \zeta^1 & = \xi^0\xi^1 + \xi^1\xi^2 + \xi^0\xi^2,\\
      \zeta^2 & = (\xi^1)^2 + \xi^0\xi^1,
    \end{align}
  \end{subequations}
  is an example of the use of the \texttt{subequations} environment.  The
  \texttt{subfigure} package defines the command
  \verb+\subfigure[+\textit{caption text}\verb+]{+\textit{figure
      name}\verb+}+.  Note that each of the subfigures may have a different
  caption.  The \texttt{subequations} environment is documented in the file
  \verb+amsldoc.dvi+, and the \texttt{subfigure} package is
  documented in \verb+subfigure.dvi+
\item[\textmd{\texttt{array.sty:}}] This package expands the column formatting
  options available in the \texttt{array} and \texttt{tabular} environments.
  For instance, columns whose entries are \textbf{\textit{vertically}}
  centered, or columns whose entries are assumed to be mathematical
  expressions are easily created.  Furthermore it allows the vertical line to
  be replaced by an arbitrary command. The package is documented in the file
  \verb+array.dvi+.
\item[\textmd{\texttt{tabularx.sty:}}] This package is an extension of the
  \texttt{array.sty} package.  It implements a version of the \texttt{tabular}
  environment in which the total width of the table is specified, and the
  widths of some columns are adjusted in order to achieve this overall width.
  Note that this is \textbf{\textit{different}} from the \texttt{tabular*}
  environment.  The package is documented in the file
  \verb+tabularx.dvi+.
\item[\textmd{\texttt{hhline.sty:}}] This package defines the command
  \verb+\hhline{+\textit{options}\verb+}+ which is similar to \verb+\hline+
  except that resulting horizontal line's interaction with vertical lines can
  be controlled by the user with the \textit{options} list.  The package is
  documented in the file \verb+hhline.dvi+.
\item[\textmd{\texttt{enumerate.sty:}}] This package redefines the standard
  \texttt{enumerate} environment so that it has an optional argument which
  specifies the style of the list counter.  The package is documented in the
  file \verb+enumerate.dvi+.
\item[\textmd{\texttt{caption.sty:}}] This package redefines the standard
  \verb+\caption{+\textit{caption text}\verb+}+ command so that the caption
  format can be customized by the user.  For instance, the package allows the
  font size used in the caption to be changed, and allows the caption text to
  be hung from the caption label.  The package is documented in the file
  \verb+caption.dvi+.  The package has an unusual bug, the length
  \verb+\captionmargin+ can \textbf{\textit{not}} be specified in inches, all
  other units appear to work.
\item[\textmd{\texttt{twocolumnconf.cls:}}] This item is actually a new
  document class written by James Howse, rather than a package.  This class
  defines a two column format which conforms to that required by many
  conferences for proceedings submissions.  This class is selected with the
  command
  \verb+\documentclass[+\linebreak\textit{options}\verb+]{twocolumnconf}+.
  The major difference between it and the standard two column format is that
  in this format the abstract spans both columns, the section headings are not
  numbered, and the pages have no numbers on them.  The title section is
  entered exactly as in \cite[pages 181--182]{Lam94a} except that the command
  \verb+\begin{abstract}+ \textit{text} \verb+\end{abstract}+ occurs
  \textbf{\textit{before}} the \verb+\maketitle+ command.  This class has the
  same \textit{options} as the standard \texttt{article} class in \cite[page
  177]{Lam94a}, \textbf{\textit{except}} the \verb+notitlepage+ $|$
  \verb+titlepage+ and \verb+onecolumn+ $|$ \verb+twocolumn+ options do not
  exist.  Also the \verb+final+ $|$ \verb+draft+ option has been modified so
  that under the \verb+draft+ option the page numbers appear at the bottom of
  each page, and under \verb+final+ the pages have no numbers.  The amount
  that the abstract text is indented is controlled by the length
  \verb+\abstractmargin+.  The default value of \texttt{0.5in} can be changed
  with the command \verb+\setlength{\abstractmargin}{+\textit{length}\verb+}+.
\item[\textmd{\texttt{unmeethesis \& unmeereport:}}] These two items are
  actually new document classes written by James Howse, rather than packages.
  The \texttt{unmeethesis} class conforms to the dissertation and thesis
  requirements of the Office of Graduate Studies at the University of New
  Mexico\footnote{Although compliance is not guaranteed, the authors'
    dissertations were accepted without difficulty.}.  The \texttt{unmeereport}
  class is intended to produce well-formated technical reports for the
  Department of Electrical Engineering at the University of New Mexico.  Both
  of these classes are documented in the file
  \verb+styles.ps+ (\textsc{PostScript}) or \verb+styles.pdf+ 
  ($\text{\textsc{Adobe}}^{\circledR}$ \texttt{PDF}), available through
  the UNM Office of Graduate Studies (OGS).
  Or alternatively, on the mirror website
  \textsf{http://www.math.unm.edu/\hspace{.9mm}$\bf\tilde{}$\hspace{.3mm}nedoren/latex} in the subdirectory \textsf{style\_sheets}\hspace{1mm}.
\end{description}
\vspace{-\parskip} Note that this list contains only a small number of the
packages available for \LaTeXe .  For the documentation of additional
packages, ask your system administrator.

There are two major issues not covered by the above list of style options.
They are the construction and inclusion of figures in \TeX\ documents, and the
use of \textsc{PostScript} fonts in \TeX\ documents.  There are two possible
ways to put figures into \TeX\ documents.  The first is to use an auxiliary
program to draw the figure and then use some package to incorporate the figure
into the document.  The second approach is to use some set of macros to draw
the figures within \TeX.  The advantage of the first approach is that it
allows almost any type of drawing to be created fairly quickly.  \TeX\ has an
\textbf{\textit{extremely}} limited ability to perform mathematical
computations, so drawing arbitrary shapes, particularly graphs of mathematical
functions, is generally very slow at best.  Also, since \TeX\ was not designed
for figure drawing, the figure must be drawn by command entry, and there is no
interactive way to see what is being drawn.  The advantage of the second
approach is that all of \TeX 's formating capabilities can be used to annotate
the figure.  This means that mathematical equations and symbols can be used in
the figure.  It is generally not possible to do much, if any, mathematical
annotation in a drawing program.  Also, in this scheme the type font of the
figure and document will be identical.  A package called \texttt{psfrag} gives
the best of both worlds by allowing \textbf{\textit{any}} text string in a
\textsc{PostScript} figure to be replaced by \textbf{\textit{any}} \LaTeXe\ 
string.  Table~\ref{tbl:ext_draw_tex} lists some public domain drawing
programs, and covers several methods for incorporating their output into \TeX\ 
documents.
%
\begin{table}[htbp]
\setlength{\topsep}{0in}
\begin{center}
\begin{tabular}{|>{\centering}m{1in}||m{5.5in}|}
\hhline{--}
\multicolumn{2}{|c|}{\rule[-0.075in]{0in}{0.25in}\Large\bf Incorporating
  External Drawings into \TeX}
\\ \hhline{|==|}
\multicolumn{2}{|c|}{\large\textit{Auxiliary Drawing Programs for \TeX}} 
\\ \hhline{--}
\textsl{Program} & \multicolumn{1}{c|}{\textsl{Description of the program}}
\\ \hhline{-||-}
{\texttt{xfig \\ tgif \\ idraw}\footnotemark} & 
All of these are interactive drawing programs.  They are all designed for free
hand type drawing.  None are suitable for high precision drawing such as CAD,
for plotting sets of data, or for plotting mathematical functions.  All are
capable of \textsc{PostScript} output.  Essentially they are all more or less
imitations of $\text{\textsf{Mac}}^{\circledR}$ drawing programs, such as
$\text{\textsf{Canvas}}^{\circledR}$, designed to work under X-Windows.  The
only documentation for any of these programs is obtained by using the command
\texttt{man} \textit{program}.  These man pages can be printed using the
command \texttt{man -t} \textit{program}.
\\ \hhline{-||-}
{\texttt{dataplot \\ gnuplot \\ xmgr \\ xprism2 \\ xprism3}\footnotemark} &
All of these programs are designed to plot lists of data points and
mathematical functions.  \texttt{xmgr}, \texttt{xprism2}, and \texttt{xprism3}
are menu-driven packages with built-in X-Windows interfaces.
\texttt{dataplot} and \texttt{gnuplot} are command-driven packages whose plots
appear in separate windows.  Both of these packages have auxiliary X-Windows
interfaces built on Tcl/Tk.  \texttt{xprism2} is for 2-dimensional plots,
while \texttt{xprism3} produces 3-dimensional plots.  \texttt{xmgr},
\texttt{dataplot}, and \texttt{gnuplot} can create either 2 or 3-dimensional
plots.  All are capable of \textsc{PostScript} output.  The only documentation
for \texttt{xmgr}, \texttt{xprism2}, and \texttt{xprism3} is that available
interactively via the \textsf{help} menu.  Both \texttt{dataplot} and
\texttt{gnuplot} have man pages and external documentation. 
For \texttt{gnuplot}, the manuals are
in the files \verb+gnuplot-3.5.dvi+, \verb+gnuplot_tutor.dvi+,
\verb+gnuplot_refcard.dvi+, and \verb+xgnuplot.doc+.
\\ \hhline{|=:b:=|}
\multicolumn{2}{|c|}{\large\textit{Packages to Incorporate Drawings into
  \TeX}}
\\ \hhline{--}
\textsl{Package} & \multicolumn{1}{c|}{\textsl{Description of the package}}
\\ \hhline{-||-}
{\texttt{epsfig.sty \\ graphicx.sty \\ graphics.sty}} & 
All of these packages are designed to incorporate \textsc{PostScript}
files into a \TeX\ \texttt{.dvi} output file.  They are entered as packages
with the command \verb+\usepackage{+\textit{package name}\verb+}+.  To enter a
centered figure into a document with \texttt{epsfig.sty}, use the command
\verb+\centerline{\psfig{\figure=+\textit{file}\verb+,+\textit{options}\verb+}}+
at the desired location in the document.  For \texttt{graphicx.sty} and
\texttt{graphics.sty} the required command is
\verb+\centerline{\includegraphics[+\textit{options}\verb+]{+\textit{file}\verb+}}+.
The documentation for all three of these packages is in 
the file \verb+grfguide.dvi+.
\\ \hhline{-||-}
\texttt{psfrag.sty} &
This package allows \textbf{\textit{any}} text string in a \textsc{PostScript}
figure to be replaced by \textbf{\textit{any}} \LaTeXe\ construction.  This
allows graphs and drawings produced by other packages to be annotated with
equations or other scientific text from \LaTeXe.  The package is invoked with
the command \verb+\usepackage{psfrag}+.  The figure text is replaced by the
\LaTeXe\ text with the command \verb+\psfrag{+\textit{figure
    text}\verb+}{+\textit{\LaTeXe\ text}\verb+}+.  In order for this command
to work the figure must be preprocessed with the command \texttt{ps2frag}
\textit{file}.  The documentation for this package is in the file
\verb+pfgguide.ps+, and there is a man page for \texttt{ps2frag}.
\\ \hhline{--}
\end{tabular}
\end{center}
\caption{Programs and packages for incorporating external drawings into \TeX.}
\label{tbl:ext_draw_tex}
\end{table}
%

The packages \texttt{epsfig}, \texttt{graphicx}, \texttt{graphics}, and
\texttt{psfrag} can only use \textsc{PostScript} files that follow the
bounding box comment convention.  This is a comment of the form
\texttt{\%\%Bounding Box:}\textit{ bbllx bblly bburx bbury}.  The four numbers
specify the lower left and upper right $x$ and $y$ coordinates in points.  If
this comment does not appear at the beginning of your \textsc{PostScript}
document, then you must specify the bounding box by hand in order to use any
of these packages.  The packages \texttt{epsfig}, \texttt{graphicx}, and
\texttt{graphics} all allow scaling, rotation, and clipping of the
\textsc{PostScript} image.  \textsc{PostScript} files incorporated using any
of these packages can be previewed with either \texttt{xdvi} or \texttt{xtex}.
Alternately the \texttt{.dvi} can be converted to \textsc{PostScript} using
\texttt{dvips}, and the resulting \texttt{.ps} file previewed using
\texttt{ghostview}, but this introduces two additional steps into document
processing.  
%::::::::
% Manually insert footnotes from table 2
%::::::::
\addtocounter{footnote}{-1}%
\footnotetext{The author prefers \texttt{xfig} to  \texttt{idraw}, and has
  very little experience with \texttt{tgif}.}%
\addtocounter{footnote}{1}%
\footnotetext{The author recommends either \texttt{gnuplot} or \texttt{xmgr}.}%
%
If your favorite drawing program does not output \textsc{PostScript}, then its
output must be converted.  The program \texttt{xv} can convert \texttt{.gif},
\texttt{.tif}, and \texttt{.jpg} formats to \textsc{PostScript}.  By the way,
this program has a number of other useful functions for image processing work.
As stated earlier, there are also packages of macros for drawing within \TeX .
These packages are briefly discussed in Table~\ref{tbl:prod_draw_tex}.
%
\begin{table}[htbp]
\setlength{\topsep}{0in}
\begin{center}
\begin{tabular}{|>{\centering}m{1in}||m{5.5in}|}
\hhline{--}
\multicolumn{2}{|c|}{\rule[-0.075in]{0in}{0.25in}\Large\bf Producing Drawings
  in \TeX}
\\ \hhline{|=:t:=|}
\textsl{Package} & \multicolumn{1}{c|}{\textsl{Description of the package}}
\\ \hhline{-||-}
{\texttt{picture \\ epic.sty \\ eepic.sty}} &
\texttt{picture} is picture drawing environment that is built in to \LaTeXe .
It is very low level and requires that explicit coordinates be specified for
\textbf{\textit{every}} object in the picture.  Also it can only draw lines of
certain slopes and circles of certain radii.  \texttt{epic} is an extension of
\texttt{picture} and \texttt{eepic} an extension of \texttt{epic}.  The
\texttt{picture} environment is documented in \cite[pages 118--129]{Lam94a}.
\\ \hhline{-||-}
{\PiCTeX \\ \Xy-pic\footnotemark} &
Both of these packages are extensive extensions of the \LaTeXe\
\texttt{picture} environment.  They are designed at allow graphs and diagrams
to be drawn inside \TeX\ and \LaTeXe.   They are both entered as packages in
\LaTeXe\ with the commands \verb+\usepackage{pictex}+ and
\verb+\usepackage[+\textit{options}\verb+]{xy}+ respectively.  The manual for
\PiCTeX\ is \textbf{\textit{not}} public domain, even though the macro files
are.  The system administrator should have a copy of the manual, which is
excellent due to the plethora of useful examples.  The documentation for
\Xy-pic\ is in the directory \verb+doc+ in the files \texttt{xyguide.dvi}
and \texttt{xyrefer.dvi}.  The \Xy-pic\ package has been kept far more up to
date by its authors than the \PiCTeX\ package.
\\ \hhline{-||-}
{\TeX draw \\ \textsc{PSTricks}\footnotemark} &
Both of these are macro packages that allow \textsc{PostScript} drawings to be
produced in \TeX .  Both of these packages basically allow a large number of
both low and high-level \textsc{PostScript} commands to be used from within a
\TeX\ or \LaTeXe\ document.  They are both entered as packages in \LaTeXe\
with the commands \verb+\usepackage{texdraw}+ and \verb+\usepackage{pstricks}+
respectively.  The documentation for both packages is in the directory
\verb+doc+.  For \TeX draw the manual is the file \texttt{texdraw.ps},
and for \textsc{PSTricks} the documentation is in the files
\texttt{pst-usr1.ps}, \texttt{pst-usr2.ps}, \texttt{pst-usr3.ps}, and
\texttt{pst-usr4.ps}.  Note that neither of these packages has been updated
for \LaTeXe , although both should work with it.
\\ \hhline{--}
\end{tabular}
\end{center}
\caption{Packages for producing drawings in \TeX.}
\label{tbl:prod_draw_tex}
\end{table}
%

The final issue to discuss is the inclusion of \textsc{PostScript} fonts in
\TeX\ documents.  If you are using \LaTeXe, then changing fonts in the main
body of the document is actually quite easy.  The general font selection
procedure is described in detail in the document \verb+fntguide.dvi+.
%::::::::
% Manually insert footnotes from table 3
%::::::::
\addtocounter{footnote}{-1}%
\footnotetext{The author feels that \PiCTeX\ and \Xy-pic\ are far superior to
  the \texttt{picture} environment and its extensions.}%
\addtocounter{footnote}{1}%
\footnotetext{The author has never used either of these packages.}%
%
This font selection scheme, which is often called \textsf{nfss}, was developed
by Mittelbach and Sch\"{o}pf for \AmSLaTeX , and was recently incorporated
into \LaTeXe .  With the introduction of this font selection mechanism,
accessing both the original \TeX\ fonts and \textsc{PostScript} fonts has
become much easier.  For instance, the \textsf{Concrete} fonts developed by
Donald Knuth are accessed with the command \verb+\usepackage{concmath}+.  The
naming convention used in \LaTeXe\ for \textsc{PostScript} fonts is described
in the document \verb+fontname.dvi+.  The \textsf{psnfss} package
provides numerous style files which redefine the default fonts to
\textsc{PostScript} fonts.  The fonts \textsf{Avant Garde}, \textsf{Bookman},
\textsf{Helvetica}, \textsf{New Century Schoolbook}, \textsf{Palatino},
\textsf{Times Roman}, \textsf{Utopia}, and \textsf{Zapf Chancery} all
currently work.  These fonts are made the default font in a document using the
commands \verb+\usepackage{avant}+, \verb+\usepackage{bookman}+,
\verb+\usepackage{helvet}+, \verb+\usepackage{newcent}+,
\verb+\usepackage{palatino}+, \verb+\usepackage{times}+,
\verb+\usepackage{utopia}+, and \verb+\usepackage{chancery}+, respectively.
The file \verb+PSFonts/ps_font_test.dvi+ contains complete character
tables and tests of the available series and shape variations for all of these
\textsc{PostScript} fonts and for all of the original \TeX\ fonts.  Note that
character tables and a great deal of other information can be obtained for
\textit{\textbf{any}} font by copying the file
\verb+/usr/local/texmf/lib/tex/latex/inputs/testdoc/nfssfont.tex+ into a
personal directory, executing the command \verb+latex nfssfont.tex+, and
following the subsequent instructions.  The real difficulty is incorporating
\textsc{PostScript} math fonts into a document.  The package
\texttt{mathtime.sty} replaces some of the normal \TeX\ math symbols with
their \textsc{PostScript} equivalents.  There are two principle problems with
using \textsc{PostScript} math symbols.  The first is that not all of the math
symbols in \LaTeXe\ and \AmSLaTeX\ are available in the public domain
\textsc{PostScript} fonts.  This means that the two sets of fonts must be
combined in order to retain all of the symbols.  The second is that the math
symbols in \TeX\ are accessed by commands of the form \verb+\+\textit{symbol}
(e.g., \verb+\gamma+, or \verb+\sum+), so these commands must all be redefined
in terms of the \textsc{PostScript} fonts.  If you want to see how this can be
done, see the file \texttt{mathtime.sty}, and \cite[pages 154--156, 357--360,
and 428--433]{Knu86a}.  The \textsf{psnfss} package is documented in the file
\verb+psnfss2e.dvi+ and in \cite[pages 332--340]{Goo94a}.  One nice
feature of the \textsf{psnfss} package is the \texttt{pifont.sty} package.
This allows the characters from the \textsf{Zapf Dingbats} font to be used as
labels in a list or elements in a line.  Note that when using
\textsc{PostScript} fonts in a document, it is best to preview it with
\texttt{ghostview} rather than \texttt{xdvi}.  For some unknown reason
\texttt{ghostview} displays ragged right margins, even when the margins are
justified as in this document.  In spite of the on-screen appearance, the
document margins are justified when printed.

If all of these hacks to get \textsc{PostScript} into \TeX\ are making you
nauseous\footnote{The author definitely falls into this category.}, there is
one other alternative.  There is a document layout program called
$\mathcal{LOUT}$, which is similar to \TeX\ in that it resembles a programming
language, whose compiled output is native \textsc{PostScript}.  Furthermore,
in the same way that \TeX\ resembles a procedural programming language,
$\mathcal{LOUT}$ resembles an object oriented programming language.
Essentially $\mathcal{LOUT}$ was written as the result of 10 years of using
\TeX\ and seeing some of it's pitfalls in hindsight.  $\mathcal{LOUT}$ was
written by Jeffrey Kingston of the Basser Department of Computer Science at
the University of Sydney.  It is available from \textsf{ftp.cs.su.oz.au} in
the directory \texttt{/jeff/lout} as \texttt{lout.3.06.tar.Z}.

In closing, here are a few \TeX\ tricks that are not in the \TeX -FAQ.  The 3
characters $\setminus$, \symbol{"5E}, and \symbol{"7E} are used in \LaTeXe\ as
parts of commands, and can not normally be printed.  One way to print them is
to use the commands \verb+$\setminus$+, \verb+\symbol{"5E}+, and
\verb+\symbol{"7E}+ respectively.  By default, \AmSLaTeX\ does
\textbf{\textit{not}} break pages in the middle of equations.  Putting the
command \verb+\allowdisplaybreaks+ after \verb+\documentclass{+$\cdot$\verb+}+
and before \verb+\begin{document}+ allows page breaks to occur in equations.
  If your document is a few lines over some page limit, insert the command
  \verb+\enlargethispage*{+\textit{length}\verb+}+ somewhere in the text of
  the what you want to be the final page.  As a rule of thumb, set
  \textit{length} to \verb+2ex+ for each line that does not fit on the page.
  The font used on the title page is selected using the command
  \verb+\fontfamily{cminch}\selectfont+.  If the chosen font size is 12 points
  or larger, then the letters are one inch tall, otherwise they are about
  three-fourths of an inch tall.  Also, if you like the tables in this
  document, read the examples in \cite[pages 204--207]{Lam94a} carefully. The
  tables were formatted using the commands shown in the following example.

\noindent\textbf{Table Example:} \LaTeXe\ code
%
\vspace{-\parskip}
\begin{verbatim}
\usepackage{array}
\usepackage{hhline}
\usepackage[sl,bf,hang,small]{caption}
\end{verbatim}
%
\vspace{-1.25\parskip}
\hspace{1em}$\vdots$
\vspace{-1.25\parskip}
%
\begin{verbatim}
\begin{table}[htbp]
\setlength{\topsep}{0in}
\begin{center}
\begin{tabular}{|>{\centering}m{1in}||m{5.5in}|}
\hhline{--}
\multicolumn{2}{|c|}{\rule[-0.075in]{0in}{0.25in}\Large\bf Incorporating
  External Drawings into \TeX}
\\ \hhline{|==|}
\multicolumn{2}{|c|}{\large\textit{Auxiliary Drawing Programs for \TeX}} 
\\ \hhline{--}
\textsl{Program} & \multicolumn{1}{c|}{\textsl{Description of the program}}
\\ \hhline{-||-}
\end{verbatim}
%
\vspace{-1.25\parskip}
\hspace{1em}$\vdots$
\vspace{-1.25\parskip}
%
\begin{verbatim}
{\texttt{epsfig.sty \\ graphicx.sty \\ graphics.sty}} & 
All of these packages are designed to incorporate \textsc{PostScript}
files into a \TeX\ \texttt{.dvi} output file.  They are entered as packages
with the command \verb+\usepackage{+\textit{package name}\verb+}+.  To enter a
centered figure into a document with \texttt{epsfig.sty}, use the command
\verb+\centerline{\psfig{\figure=+\textit{file
      name}\verb+,+\textit{options}\verb+}}+ at the desired location
in the document.  For \texttt{graphicx.sty} and \texttt{graphics.sty} the
required command is
\verb+\centerline{\includegraphics[+\textit{options}\verb+]{+\textit{file
      name}\verb+}}+.  The documentation for all three of these packages is in
the file \verb+grfguide.dvi+.
\\ \hhline{-||-}
\end{verbatim}
%
\vspace{-1.25\parskip}
\hspace{1em}$\vdots$
\vspace{-1.25\parskip}
%
\begin{verbatim}
\\ \hhline{--} 
\end{tabular}
\end{center}
\caption{Programs and packages for incorporating external drawings into \TeX.}
\label{tbl:ext_draw_tex}
\end{table}
\end{verbatim}
%
Note that the column specifier \verb+m{+\textit{length}\verb+}+ vertically
centers the two adjacent columns.  The prefix command \verb+>{\centering}+
horizontally centers the entries in the first column.  Also note the use of
the \verb+\multicolumn+ environment spanning only one column in order to
center the header for column two.  The vertical bar only needs to appear on
the right of this specifier because the original declaration following
\verb+\begin{tabular}+ inserts it on the left side.  The enumerated list was
formatted using the commands shown in the following example.

\noindent\textbf{List Example:} \LaTeXe\ code
%
\vspace{-\parskip}
\begin{verbatim}
\usepackage{enumerate}

\newlength{\enumerateindent}
\setlength{\enumerateindent}{\leftmargini}
\addtolength{\enumerateindent}{-\labelsep}
\addtolength{\enumerateindent}{-0.4\labelwidth}
\end{verbatim}
%
\vspace{-1.25\parskip}
\hspace{1em}$\vdots$
\vspace{-1.25\parskip}
%
\begin{verbatim}
\begin{enumerate}[\hspace{\enumerateindent}1) ]
\item Seven new alignment formats for equations.  For example, these alignment
  formats allow the following two constructions which would be rather
  difficult in \LaTeXe.  The first construction is
\end{verbatim}
%
\vspace{-1.25\parskip}
\hspace{1em}$\vdots$
\vspace{-1.25\parskip}
%
\begin{verbatim}
\end{enumerate}
\end{verbatim}
%
Notice that in order to indent the elements of the list, a space
\textbf{\textit{must}} be explicitly inserted.  One last example which may be
useful is an example of putting a double box around a title page using
\texttt{fancybox.sty}.  Note that this set of commands is
\textbf{\textit{not}} taken from this document.

\noindent\textbf{Box Example:} \LaTeXe\ code
%
\vspace{-\parskip}
\begin{verbatim}
\usepackage{fancybox}
\end{verbatim}
%
\vspace{-1.25\parskip}
\hspace{1em}$\vdots$
\vspace{-1.25\parskip}
%
\begin{verbatim}
\begin{document}

\thisfancyput*(-0.1in,-9.1in){%
  \setlength{\unitlength}{1in}%
  \linethickness{2\fboxrule}%
  \framebox(6.7,9.2){%
    \linethickness{\fboxrule}%
    \framebox(6.6,9.1)\protect\rule{0in}{0pt}%
    }%
  }
\title{{\fontfamily{cminch}\selectfont \TeXTiPs\\[0.5ex]} Supplementary
  information about \LaTeXe}
\author{James W. Howse}
\maketitle
\end{verbatim}
%
The command \verb+\protect\rule{0in}{0pt}+ is an ``invisible'' character which
insures that the two boxes are both vertically and horizontally centered with
respect to each other.  Note that the position of the box is determined by the
location of the lower-left hand corner.  If you are curious about the commands
used to achieve any other formatting in this document, see the file
\verb+tex_tips.tex+.

\vspace{8mm}
\noindent
{\bf \Large Contacting the Author}

\noindent
This document is maintained by Neall Doren.  He can be reached via email
at \texttt{nedoren@sandia.gov}\,.  Please report any anomalies or
suggestions for changes to Dr.\ Doren, via email only.
The document you are reading is
is available via the OGS website.  As an alternative, it can be downloaded from
the mirror website
\textsf{http://www.math.unm.edu/\hspace{.9mm}$\bf\tilde{}$\hspace{.3mm}nedoren/latex} in the subdirectory \textsf{user\_manuals}
under filename \textsf{textips.ps} (\textsc{PostScript}) or
\textsf{textips.pdf}
($\text{\textsc{Adobe}}^{\circledR}$ \texttt{PDF}).


%
\bibliographystyle{alpha}
\bibliography{%
/research/mesquite/howse/write/Bibtex/howse_doc_prep%
}
\end{document}

% LocalWords:  pt titlepage amsmath amsbsy amsopn amsthm upref amsfonts mathscr
% LocalWords:  eucal eucurs epsfig psfrag hhline sl bf fancyheadings fancybox
% LocalWords:  emP emC emu emc pic cminch Howse tex dvi Horne htbp dvips ps ftp
% LocalWords:  PostScript xdvi xtex WSYWIG ghostview sunsite auc dk auctex FAQ
% LocalWords:  comp rtfm mit edu usenet faq sty endfloat theapa bst APA cls Tcl
% LocalWords:  setspace baselinestretch subequations subfigure subequation xfig
% LocalWords:  tabularx twocolumnconf unmeethesis unmeereport tgif idraw xmgr
% LocalWords:  dataplot gnuplot xprism Tk graphicx frag bbllx bblly bburx bbury
% LocalWords:  gif tif jpg eepic xyguide xyrefer PSTricks texdraw pst usr nfss
% LocalWords:  Mittelbach Schopf psnfss Garde Bookman Palatino Zapf mathtime cs
% LocalWords:  pifont Dingbats Basser su oz au jeff lout ICS Sch pf
