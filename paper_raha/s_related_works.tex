\section{Related Works}	\label{sec:related_works}

Works related to MABDI are generally based on RGB-D sensors. This type of sensor has
become very popular since the release of the Kinect from Microsoft, which
was the first mass produced RGB-D sensor of its kind. RGB-D sensors are
inexpensive and produce noisy 640x480 depth images at 30fps. The RGB-D
sensor has excited the robotics community because this has been the first
time that depth data has been so readily accessible from such an
inexpensive sensor. Therefore, methodologies that use RGB-D data must be able to quickly
deal with very high rates of information.

One highly-related work was published by Whelan
et al. in 2012 \cite{Whelan2012} and more recently in 2013
\cite{Whelan12tr}. The system they developed was named Kintinuous and was
able to produce a high quality mesh representation of the environment.
Their hybrid system utilized the KinectFusion method
\cite{Newcombe2011a} of Newcombe et al. to create a volumetric
representation of the portion of the environment in front of the sensor. As
the sensor moves, portions of the environment that leave the volume in
front of the sensor are ray cast and turned into a mesh. They obtain very
impressive results but also mention a limitation of their system for future
work. The limitation is that the mesh can not be updated once created,
which is an issue when revisiting parts of the environment. One of the most
impressive current works which has an adaptable mesh came from Cashier et
al. in 2012 \cite{Cahier2012}. In this work, they were able to generate and
update a mesh with new measurements from a ToF sensor. They used the
difference between the existing model and the actual measurements to decide
whether to adapt the mesh or add new elements. The mesh topology was not
adaptive to the environment and their experiments only showed results of mapping a
single flat wall with no robot movement. The system needs to be tested for
object addition and removal.

Research and development of new mapping algorithms trend towards
leveraging the information in the global map to make decisions about the
incoming data. One can see parallels with how we as humans see the world. MABDI
proposes do this in a computationally feasible way by simply using
differencing and thresholding imaging methods.








% One very impressive work came from Henry et al. in 2012 \cite{Henry2012}. In
% this work they designed a system which used a RGB-D sensor to build a map made
% of surfels (Surfels are circular disks which have a particular position and
% orientation and also a radial size based on confidence.). In order to generate
% and maintain the surfel map they used the work of Weise et al. \cite{Weise2009}.
% The map consists of a large number of surfels. The surfel map can be updated
% given new registered depth images from the sensor. Decisions are made how
% to handle each measurement in the depth image based on the difference between an
% expectation generated using the current map and the actual readings from the
% sensor. Rendering a surfel map requires special methods \cite{Pfister2000} and
% is difficult to use in applications such as obstacle avoidance.
%
% One of the next major advances was published by Whelan
% et al. in 2012 \cite{Whelan2012} and more recently in 2013
% \cite{Whelan12tr}. The system they developed was named Kintinuous and was
% able to produce a high quality mesh representation of the environment.
% Their hybrid system utilized the KinectFusion method
% \cite{Newcombe2011a} of Newcombe et al. to create a volumetric
% representation of the portion of the environment in front of the sensor. As
% the sensor moves, portions of the environment that leave the volume in
% front of the sensor are ray cast and turned into a mesh. They obtain very
% impressive results but also mention a limitation of their system for future
% work. The limitation is that the mesh can not be updated once created,
% which is an issue when revisiting parts of the environment. One of the most
% impressive current works which has an adaptable mesh came from Cashier et
% al. in 2012 \cite{Cahier2012}. In this work, they were able to generate and
% update a mesh with new measurements from a ToF sensor. They used the
% difference between the existing model and the actual measurements to decide
% whether to adapt the mesh or add new elements. The mesh topology was not
% adaptive to the environment and their experiments only showed results of mapping a
% single flat wall with no robot movement. The system needs to be tested for
% object addition and removal.
