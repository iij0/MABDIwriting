\section{Conclusion} \label{sec:conclusion}

The goal of MABDI is to determine data from the sensor that has not yet been
represented in the map and use this data to add to the map. MABDI does this by
leveraging the difference between what we are actually seeing and what we expect
to see. MABDI can work in conjunction with any ``black box'' mesh-based surface
reconstruction algorithm, and can be thought of as a general means to provide
introspection to those types of reconstruction methods.

 The MABDI implementation was able to successfully perform in a realistic
 simulation environment. The results show how novel sensor data was
 successfully classified and used to add to the global mesh. Also, the MABDI
 algorithm runs at around 2Hz on a consumer grade laptop with an Intel i7
 processor. This performance means that it is capable of real-world
 applications.

 Currently MABDI is only designed to handle object addition, but the idea can be
 extended to handle both object addition and removal as discussed in Section
 \ref{sec:approach}. This would give the system the capability to handle highly
 dynamic environments such as a door opening and closing.
